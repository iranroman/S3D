\documentclass[14pt]{extarticle} 
\usepackage{authblk, helvet, amsmath, amsthm, amssymb, calrsfs, wasysym, verbatim, bbm, color, graphicx, geometry, hyperref, lscape, makecell, longtable, tikz, tikz-3dplot}
\usepackage[binary-units=true]{siunitx}
\renewcommand{\familydefault}{\sfdefault}

\usetikzlibrary{3d}
\definecolor{dgreen}{RGB}{63,127,0}
\definecolor{dred}{RGB}{144,14,3}

\geometry{tmargin=.75in, bmargin=.75in, lmargin=.75in, rmargin = .75in}  

\newcommand{\R}{\mathbb{R}}
\newcommand{\C}{\mathbb{C}}
\newcommand{\Z}{\mathbb{Z}}
\newcommand{\N}{\mathbb{N}}
\newcommand{\Q}{\mathbb{Q}}
\newcommand{\Cdot}{\boldsymbol{\cdot}}

\newtheorem{thm}{Theorem}
\newtheorem{defn}{Definition}
\newtheorem{conv}{Convention}
\newtheorem{rem}{Remark}
\newtheorem{lem}{Lemma}
\newtheorem{cor}{Corollary}


\title{Virtual microphone array: experimental design}
\author{Iran R. Roman \thanks{roman@nyu.edu}}
\affil{Music and Audio Research Laboratory, New York University}
\date{}

\begin{document}
\tdplotsetmaincoords{60}{50}

\maketitle
\tableofcontents

\vspace{.25in}


\section{Expected contributions}

\begin{itemize}

\item Tool to aggregate and process multiple microphone array datasets with waveform and relative location information.

\item Method to process audio data with high spatial dimensionality

\item Generalized model of spatial audio filter estimation

\end{itemize}

\section{Architectures}

\begin{itemize}

\item U-Net \cite{gao20192}.

\end{itemize}

\section{Dataset}

\begin{itemize}

\item Aggregate of DCASE challenge task 3 datasets for three different years \cite{adavanne2019multi, politis2020dataset, politis2021tau}, 3D-MARCo dataset \cite{lee3d}, and EigenScape dataset \cite{green2017eigenscape}

\item Formatted like the FAIR-Play Dataset \cite{gao2019visualsound}.

	\begin{itemize}
	\item 10 second clips
	\item Resampled to be \SI{16}{\kilo\hertz}.
	\item Randomly separated into 10 splits of equal size.
	\item 8 of the 10 splits are used for training, one for validation, and one for testing. 
	\end{itemize}


\end{itemize}

\section{Input data}

\begin{itemize}

\item STFT computed using a Hann window of length \SI{25}{\milli\second} (400 samples).

\item Hop length of 10 \SI{10}{\milli\second} (160 samples).

\item FFT size of 512. 

\item Training data consists of randomly sampled segments of \SI{0.63}{\second} from each \SI{10}{\second} `clip'. 

\item Each segment is normalized to a constant RMS value (How did \cite{gao2019visualsound} calculate this?)

\item The input is the ``complex spectrogram" with the real and imaginary part in separate channels. 

\item Testing data is a `clip' processed with a sliding window with hop size \SI{0.05}{\second}.

\item Each input segment has a corresponding set of coordinates for each channel's physical location.

\end{itemize}

\section{Experiments}

\begin{itemize}

\item Input formats: FFT, MFCC, raw waveform

\item Number of input channels: 1, 2, 3 (consistent chan depth after 1st conv). 

\begin{tikzpicture}[tdplot_main_coords,line join=miter,font=\sffamily]
\path[tdplot_screen_coords] (-2,0); 

\pgfmathsetmacro{\xstretch}{4}
\foreach \XX in {1}
{\begin{scope}[canvas is yz plane at x=\xstretch*0+0.4*\XX]
\pgfmathtruncatemacro{\fullness}{120-20*\XX}
\draw[fill=red!\fullness] (-2.5,-2.5) rectangle (2.5,2.5);
\end{scope}}
\node[anchor=west] at (\xstretch*0,2.6,2.6){$T\times F\times 1$};

\foreach \XX in {1}
{\begin{scope}[canvas is yz plane at x=\xstretch*1+0.4*\XX]
\pgfmathtruncatemacro{\fullness}{120-20*\XX}
\draw[fill=green!\fullness] (-1,-1) rectangle (1,1);
\end{scope}}
\node[anchor=west] at (\xstretch*1,1.1,1.1){$k\times k\times 1$};

\foreach \XX in {1}
{\begin{scope}[canvas is yz plane at x=\xstretch*2+0.4*\XX]
\pgfmathtruncatemacro{\fullness}{120-20*\XX}
\draw[fill=blue!\fullness] (-2.5,-2.5) rectangle (2.5,2.5);
\end{scope}}
\node[anchor=west] at (\xstretch*2,2.6,2.6){$T\times F\times 1$};
\end{tikzpicture}

\begin{tikzpicture}[tdplot_main_coords,line join=miter,font=\sffamily]
\path[tdplot_screen_coords] (-2,0); 

\pgfmathsetmacro{\xstretch}{4}
\foreach \XX in {1,2,3}
{\begin{scope}[canvas is yz plane at x=\xstretch*0+0.4*\XX]
\pgfmathtruncatemacro{\fullness}{120-20*\XX}
\draw[fill=red!\fullness] (-2.5,-2.5) rectangle (2.5,2.5);
\end{scope}}
\node[anchor=west] at (\xstretch*0,2.6,2.6){$T\times F\times 3$};

\foreach \XX in {1,2,3}
{\begin{scope}[canvas is yz plane at x=\xstretch*1+0.4*\XX]
\pgfmathtruncatemacro{\fullness}{120-20*\XX}
\draw[fill=green!\fullness] (-1,-1) rectangle (1,1);
\end{scope}}
\node[anchor=west] at (\xstretch*1,1.1,1.1){$k\times k\times 3$};

\foreach \XX in {1}
{\begin{scope}[canvas is yz plane at x=\xstretch*2+0.4*\XX]
\pgfmathtruncatemacro{\fullness}{120-20*\XX}
\draw[fill=blue!\fullness] (-2.5,-2.5) rectangle (2.5,2.5);
\end{scope}}
\node[anchor=west] at (\xstretch*2,2.6,2.6){$T\times F\times 1$};
\end{tikzpicture}

\item Adding channel coordinates to each input feature.  

\item Convolutions over time X frequency vs time X channels vs time X frequency X channel 

\end{itemize}


\bibliography{references}
\bibliographystyle{plain}

\end{document}
