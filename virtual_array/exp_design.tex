\documentclass[14pt]{extarticle} 
\usepackage{authblk, helvet, amsmath, amsthm, amssymb, calrsfs, wasysym, verbatim, bbm, color, graphicx, geometry, hyperref, lscape, makecell, longtable}
\usepackage[binary-units=true]{siunitx}
\renewcommand{\familydefault}{\sfdefault}

\geometry{tmargin=.75in, bmargin=.75in, lmargin=.75in, rmargin = .75in}  

\newcommand{\R}{\mathbb{R}}
\newcommand{\C}{\mathbb{C}}
\newcommand{\Z}{\mathbb{Z}}
\newcommand{\N}{\mathbb{N}}
\newcommand{\Q}{\mathbb{Q}}
\newcommand{\Cdot}{\boldsymbol{\cdot}}

\newtheorem{thm}{Theorem}
\newtheorem{defn}{Definition}
\newtheorem{conv}{Convention}
\newtheorem{rem}{Remark}
\newtheorem{lem}{Lemma}
\newtheorem{cor}{Corollary}


\title{Virtual microphone array: experimental design}
\author{Iran R. Roman \thanks{roman@nyu.edu}}
\affil{Music and Audio Research Laboratory, New York University}
\date{}

\begin{document}

\maketitle
\tableofcontents

\vspace{.25in}

\section{Dataset}

\begin{itemize}

\item Formatted like the FAIR-Play Dataset \cite{gao2019visualsound}.

\item 10 second clips

\item Resampled to be \SI{16}{\kilo\hertz}.

\item Randomly separated into 10 splits of equal size.

\item 8 of the 10 splits are used for training, one for validation, and one for testing. 

\end{itemize}

\section{Input data}

\begin{itemize}

\item STFT computed using a Hann window of length \SI{25}{\milli\second} (400 samples).

\item Hop length of 10 \SI{10}{\milli\second} (160 samples).

\item FFT size of 512. 

\item Training data consists of randomly sampled segments of \SI{0.63}{\second} from each \SI{10}{\second} clip. 

\item Each segment is normalized to a constant RMS value. 

\item The input is the ``complex spectrogram" with the real and imaginary part in separate channels. 

\item Testing data is a clip processed with a sliding window with hop size \SI{0.05}{\second}.

\item Each input segment has a corresponding set of coordinates for each channel's physical location.

\end{itemize}

\section{Experiments}



\bibliography{references}
\bibliographystyle{plain}

\end{document}
