\documentclass[14pt, oneside]{extarticle} 
\usepackage{authblk, helvet, amsmath, amsthm, amssymb, calrsfs, wasysym, verbatim, bbm, color, graphics, geometry}
\renewcommand{\familydefault}{\sfdefault}

\geometry{tmargin=.75in, bmargin=.75in, lmargin=.75in, rmargin = .75in}  

\newcommand{\R}{\mathbb{R}}
\newcommand{\C}{\mathbb{C}}
\newcommand{\Z}{\mathbb{Z}}
\newcommand{\N}{\mathbb{N}}
\newcommand{\Q}{\mathbb{Q}}
\newcommand{\Cdot}{\boldsymbol{\cdot}}

\newtheorem{thm}{Theorem}
\newtheorem{defn}{Definition}
\newtheorem{conv}{Convention}
\newtheorem{rem}{Remark}
\newtheorem{lem}{Lemma}
\newtheorem{cor}{Corollary}


\title{A review of existings datasets with microphone array recordings}
\author{Iran R. Roman \thanks{roman@nyu.edu}}
\affil{Music and Audio Research Laboratory, New York University}
\date{March 2021}

\begin{document}

\maketitle
\tableofcontents

\vspace{.25in}

\section{Introduction}

\subsection{Background}

\begin{itemize}

\item Object localization and tracking allow intelligent agents to determine whether something is moving, in which direction, and how fast. 

\item From an evolutionary standpoint, object localization and tracking are essential for survival \cite{heffner2018evolution}.

\item Sound localization and tracking are fundamental problems of machine listening technologies.

\item Without these capabilities, {\it listening machines} cannot locate living beings and objects by the sounds they make. 

\end{itemize}

\subsection{Problem statement}

\begin{itemize}

\item Current machine listening models learn sound localization and tracking using a specific microphone array configuration. 

\item To add or remove channels in the data, the model architecture must be changed and retrained, at least at the level of the input layer.

\end{itemize}

\subsection{Proposed solution}

\begin{itemize}

\item Microphone arrays come in all sizes and shapes \cite{kurz2015comparison, bates2017comparing, lopez2019sphear}. 

\item We should not think of individual microphone array configurations as mutually exclusive.

\item Instead, we can think of them as a subset of an idealized spherical array with an infinite number of channels.

\item The idealized spherical array contains microphones at all positions within the sphere and facing all possible directions of arrival.

\item For each microphone in an existing array, we know its signal captured, position, directionality, and frequency response. 

\item By combining information across microphone arrays, we can interpolate and approximate the idal spherical array.

\end{itemize}

\subsection{Scope of this review}

\being{itemize}

\item First, we describe the importance of microphone arrays for 3D audio.

\item Then, we review the technical specifications of commonly used microphone arrays.

\item Next, we present an overview of existing datasets with microphone arrays.

\item Finally, we discuss methodologies to exploit these datasets with the goal of approximating the idealized spherical array.  

\end{itemize}

\bibliography{references}
\bibliographystyle{plain}


\end{document}
