\documentclass[14pt, oneside]{extarticle} 
\usepackage{authblk, helvet, amsmath, amsthm, amssymb, calrsfs, wasysym, verbatim, bbm, color, graphics, geometry, siunitx, hyperref}
\renewcommand{\familydefault}{\sfdefault}

\geometry{tmargin=.75in, bmargin=.75in, lmargin=.75in, rmargin = .75in}  

\newcommand{\R}{\mathbb{R}}
\newcommand{\C}{\mathbb{C}}
\newcommand{\Z}{\mathbb{Z}}
\newcommand{\N}{\mathbb{N}}
\newcommand{\Q}{\mathbb{Q}}
\newcommand{\Cdot}{\boldsymbol{\cdot}}

\newtheorem{thm}{Theorem}
\newtheorem{defn}{Definition}
\newtheorem{conv}{Convention}
\newtheorem{rem}{Remark}
\newtheorem{lem}{Lemma}
\newtheorem{cor}{Corollary}


\title{A review of existings datasets with microphone array recordings}
\author{Iran R. Roman \thanks{roman@nyu.edu}}
\affil{Music and Audio Research Laboratory, New York University}
\date{March 2021}

\begin{document}

\maketitle
\tableofcontents

\vspace{.25in}

\section{Introduction}

\subsection{Background}

\begin{itemize}

\item Localization and tracking allow intelligent agents to determine if objects are moving, in which direction, and how fast. 

\item From an evolutionary standpoint, object localization and tracking are essential for survival \cite{heffner2018evolution}.

\item Sound localization and tracking are fundamental problems of machine listening technologies.

\item Without these capabilities, {\it listening machines} cannot locate living beings and objects in their environment by the sounds they make. 

\end{itemize}

\subsection{Problem statement}

\begin{itemize}

\item Current machine listening models learn sound localization and tracking using a specific microphone array configuration. 

\item To add or remove channels in the data, the model architecture must be changed and retrained, at least at the level of the input layer.

\end{itemize}

\subsection{Proposed solution}

\begin{itemize}

\item Microphone arrays come in all sizes and shapes \cite{kurz2015comparison, bates2017comparing, lopez2019sphear}. 

\item We should not think of individual microphone array configurations as mutually exclusive.

\item Instead, we can think of them as a subset of a virtual with an infinite number of channels.

\item The virtual array is spherical in shape, and contains microphones at all positions and facing all possible directions of arrival.

\item For each microphone in an existing array, we know its signal captured, position, directionality, and frequency response. 

\item By combining information across microphone arrays, we can interpolate and approximate the virtual array.

\end{itemize}

\section{Scope of this review}

\begin{itemize}

\item This document has two goals:
	\begin{itemize}
	\item To identify existing audio datasets collected with microphone arrays. 
	\item To discuss whether they can be used to approximate the virtual array. 
	\end{itemize}

\item First, we describe the importance of microphone arrays for object localization and tracking.

\item Then, we review the technical specifications of commonly used microphone arrays.

\item Next, we present an overview of existing datasets with microphone arrays.

\item Finally, we discuss methodologies to exploit these datasets with the goal of approximating the virtual array.  

\end{itemize}

\section{Object localization and tracking with microphone arrays}

\subsection{Auditory object localization and tracking}

\begin{itemize}

\item Intelligent organisms rely on sensory cues to interact with objects in their environment. 

\item Several mammalian species can localize and track objects based on the sounds they make.

\item To do this, many animals rely on their ears (two auditory inputs) and mechanisms of \cite{grothe2010mechanisms}:
	\begin{itemize}
	\item Monoaural cues (spectral analysis)
	\item Interaural time difference
	\item Interaural level difference
	\end{itemize}

\item Evolution also gave rise to other mechanisms for sound localization and tracking, including \cite{grothe2010mechanisms}: 
	\begin{itemize}
	\item Particle motion detection in fish 
	\item Processing of interfering patterns reaching the tympanum in frogs and birds
	\end{itemize}

\end{itemize}

\subsection{The need for microphone arrays}

\begin{itemize}

\item Many mammals are able to localize and track auditory objects in their enviroment using only two ears.

\item While a microphone array can have just two microphones, most have four or more. 

\item Why do we need microphone arrays to carry out the same task that many animals carry out with two ears?

\item There two main technical problems with binaural source localization in machines \cite{kim2015improved}:
	\begin{itemize}
	\item Diffraction of sound waves around the agent's head
	\item Front back ambiguity
	\end{itemize}

\item To overcome these issues, it is necessary to move the ears and head (like humans naturally do) and filter inputs with specific head-related transfer functions.

\item In other words, the machine has to listen more like a biological agent.
	
\item From an engineering stand-point, the issue can be solved using microphone arrays, which add spatial and auditory information. 

\item In machine listening, microphone arrays have been shown to be the optimal solution to carrying out sond source localization and tracking.

\item In general, the larger the number of microphones in an array, the better sound source localization results \cite{valin2007robust}.

\item Microphone subsampling

\item This review is limited in scope to the last ? years

\end{itemize}

\section{Commonly used microphone arrays}

\section{Existing datasets with microphone arrays (by recording environment)}

\subsection{Anechoic or semi-anechoic room}

\subsubsection{Pyramic (Scheibler, 2018)}

\begin{itemize}

\item Sine sweeps, noise, and speech samples, recorded with a 48-channel microphone array in an anechoic chamber \cite{scheibler2018pyramic}

\item Hardware: the Pyramic microphone \cite{scheibler2018pyramic}
	\begin{itemize}
	\item 6 PCBs, each with 8 Micro-Electro-Mechanical System (MEMS) microphones 
	\item each PCB is \SI{27}{\centi\metre} long
	\item six microphones spaced by \SI{40}{\milli\metre}, the remaining two by \SI{8}{\milli\metre}
	\item spatial aliasing avoided up to \SI{21}{\kilo\hertz}
	\item each PCB has an ADC
	\item ADC sampling \SI{48}{\kilo\hertz} at 16 bits
	\item \SI{1.143}{\metre} height
	\end{itemize}

\item Environment: anechoic chamber
	
\item Signals
	\begin{itemize}
	\item linear and exponential sweeps
	\item noise sequence
	\item male (2) and female (3) speech
	\end{itemize}

\item Sound sources
	\begin{itemize}
	\item Genelec 8020A speaker
	\item distance \textbackslash \SI{4}{\metre}
	\item three different heights
	\item 180 uniformly spaced angles by \SI{5}{\degree}
	\end{itemize}

\item DOI: \href{10.5281/zenodo.1209563}{https://zenodo.org/record/1209563} 

\item Files available
	\begin{itemize}
	\item compressed recordings in TTA format
	\item segmented recordings
	\item all microphone's impulse responses
	\item documentation and code
	\item also available: raw wav files
	\end{itemize}

\end{itemize}

\subsection{Controlled reverberant room}

\subsubsection{3D-MARCo (Lee and Johnson, 2019)}

\begin{itemize}

\item Recordings of musical performances and room impulse reponses with different 3D microphone arrays \cite{lee3d}

\item Hardware: 71 microphones across different microphone arrays used simultaneously
	\begin{itemize}
	\item ADC sampling \SI{96}{\kilo\hertz} at 24 bits
	\item OCT-3D \cite{theile20123d}
		\begin{itemize}
		\item dimensions: \SI{1}{\metre}\texttimes\SI{0.4}{\metre}\texttimes\SI{1}{\metre}
		\item 6 supercardioid mics
		\item 3 cardioid mics
		\end{itemize}
	\item 2L-Cube 
		\begin{itemize}
		\item dimensions: \SI{1}{\metre}\texttimes\SI{1}{\metre}\texttimes\SI{1}{\metre}
		\item 9 omnidirectional mics
		\end{itemize}
	\item Decca Tree (9 omnidirectional)
		\begin{itemize}
		\item dimensions: \SI{2}{\metre}\texttimes\SI{2}{\metre}\texttimes\SI{1}{\metre}
		\item 9 omnidirectional mics
		\end{itemize}
	\item PCMA-3D \cite{lee2014effect}
		\begin{itemize}
		\item dimensions: \SI{1}{\metre}\texttimes\SI{1}{\metre}\texttimes\SI{0}{\metre}
		\item 4 supercardioid mics (up-facing)
		\item 5 cardioid mics (\SI{30}{\degree})
		\end{itemize}
	\item Hamasaki Square
		\begin{itemize}
		\item dimensions: \SI{2}{\metre}\texttimes\SI{2}{\metre}\texttimes\SI{1}{\metre}
		\item 4 fig-8 mics (side-facing)
		\item 8 cardioid mics (back-facing)
		\end{itemize}
	\item Eigenmike
		\begin{itemize}
		\item 32 capsules on a sphere
		\item HOA encoding
		\item ADC sampling \SI{48}{\kilo\hertz} at 16 bits
		\end{itemize}
	\item \textbackslash \SI{2.5}{\metre} height
	\item 250+ different front-rear-height combinations
	\end{itemize}

\item Environment: St. Paul's concert hall (RT60 \textbackslash \SI{2.1}{\second}; Huddersfield, UK)

\item Signals
	\begin{itemize}
	\item musical performances (\textbackslash 10 hours of data)
	\item sine sweeps
	\end{itemize}

\item Sound sources
	\begin{itemize}
	\item string quartet
	\item piano trio
	\item piano solo
	\item organ
	\item clarinet solo
	\item vocal group
	\item speakers
		\begin{itemize}
		\item 9 Genelec 8040A
		\item 4 Genelec 1029As
		\item 13 equally spaced angles by \SI{15}{\degree}
		\end{itemize}
	\end{itemize}

\item DOI: \href{10.5281/zenodo.3477602}{https://zenodo.org/record/3477602} 

\item Files available
	\begin{itemize}
	\item musical performances (wav)
	\item room impulse responses (wav)
	\item documentation
	\item also available: similar dataset of room impulse responses \cite{lee2017microphone}
	\end{itemize}

\end{itemize}

\subsubsection{METU SPARG (Olgun and Hacihabiboglu, 2019)}

\begin{itemize}

\item Impulse reponse measurements with the Eigenmike em32 and the Alctron M6 microphones \cite{olgun2019metu}

\item Hardware
	\begin{itemize}
	\item Eigenmike
		\begin{itemize}
		\item 32 capsules on a sphere
		\end{itemize}
	\item Alctron M6
		\begin{itemize}
		\item single channel	
		\end{itemize}
	\end{itemize}

\item Environment: classroom at METU Graduate School of Informatics
	\begin{itemize}
	\item dimensions: \SI{6.5}{\metre}\texttimes\SI{8.3}{\metre}\texttimes\SI{2.9}{\metre}
	\item RT60 \SI{1.12}{\second} 
	\end{itemize}

\item Signal: logarithmic sine sweeps

\item Sound sources
	\begin{itemize}
	\item Genelec 6010A
	\item position and height changed on a rectilinear grid surrounding the microphone(s)
	\end{itemize}

\item DOI: \href{10.5281/zenodo.2635758}{https://zenodo.org/record/2635758} 

\item Files available for each microphone
	\begin{itemize}
	\item raw recordings in wav format
	\item documentation
	\end{itemize}

\end{itemize}

\subsubsection{Detmold (Amengual et al., 2020)}

\begin{itemize}

\item Spatial room impulse responses from three performance stages, recorded with a microphone array and an artificial head \cite{amengual2020open}

\item Hardware
	\begin{itemize}
	\item open array with six NTI M2010 mics
		\begin{itemize}
		\item 6 omnidirectional microphones 
		\item 3 pairs on orthogonal axes	
		\item dimensions: \SI{10}{\centi\metre} on-axis separation
		\end{itemize}
	\item Neumann KU100 binaural head
	\item measurements at multiple locations across the room
	\end{itemize}

\item Environment
	\begin{itemize}
	\item Brahmssaal: a music chamber hall (\SI{750}{\cubic\metre})
	\item Detmold Sommertheater: a small theater with balconies (\SI{2700}{\cubic\metre})
	\item Detmold Konzerthaus: large performance room (\SI{4600}{\cubic\metre})
	\end{itemize}

\item Signal: logarithmic sine sweeps

\item Sound sources
	\begin{itemize}
	\item Neumann KH120 speaker
	\item Multiple locations on stage
	\end{itemize}

\item DOI: \href{10.5281/zenodo.4116247}{https://zenodo.org/record/4116247} 

\item Files available for each microphone
	\begin{itemize}
	\item raw recordings in wav format
	\item documentation
	\end{itemize}

\end{itemize}

\subsection{Room with ambient noise}

\subsubsection{LOCATA (Evers et al., 2020)}

\begin{itemize}

\item Recordings of stationay and moving talkers or loudspeakers using stationary or moving microphone arrays \cite{evers2020data}

\item Hardware
	\begin{itemize}
	\item DICIT microphone array \cite{brutti2010woz}
		\begin{itemize}
		\item planar array with 15 microphones 
		\item nested in a linear uniform fashion	
		\item dimensions: \SI{2.24}{\metre} length
		\end{itemize}
	\item Eigenmike
		\begin{itemize}
		\item 32 capsules on a sphere 
		\item dimensions: \SI{84}{\milli\metre} diameter
		\end{itemize}
	\item Robot head
		\begin{itemize}
		\item 12 microphones on a pseudo-sphere 
		\end{itemize}
	\item Hearing aids dummies
		\begin{itemize}
		\item mounted on a dummy head
		\item two microphones per dummy
		\item separated by \SI{9}{\milli\metre}
		\end{itemize}
	\item OptiTrack system for ground-truth positional labeling
	\end{itemize}

\item Environment: computing laboratory at Humboldt University
	\begin{itemize}
	\item dimensions: \SI{7.1}{\metre}\texttimes\SI{9.8}{\metre}\texttimes\SI{3}{\metre}
	\item RT60 \SI{0.55}{\second} 
	\item ambient road noise in front of the building
	\end{itemize}

\item Signal: speech utterances from the CSTR VCTK dataset

\item Sound sources
	\begin{itemize}
	\item Static Genelec loudspeakers (1029A and 8020C)
	\item loudspeakers played speech individually or simultaneously
	\item Moving human talkers
	\item talkers recited speech individually or simultaneously
	\end{itemize}

\item DOI: \href{10.5281/zenodo.3630471}{https://zenodo.org/record/3630471} 

\item Files available
	\begin{itemize}
	\item raw recordings in wav format
	\item ground truth positional information
	\item documentation
	\end{itemize}

\end{itemize}

\subsection{Real-world environments}

\subsubsection{DEMAND (Thiemann et al., 2013)}

\begin{itemize}

\item Multi-channel recordings of acoustic noise in diverse environments \cite{thiemann2013demand}

\item Hardware
	\begin{itemize}
	\item Custom microphone array
		\begin{itemize}
		\item 16 omnidirectional electret condenser mics (Sony ECM-C10) 
		\item placed on a staggered grid pattern 
		\item microphone separation: \SI{5}{\centi\metre}
		\end{itemize}
	\end{itemize}

\item Environment
	\begin{itemize}
	\item domestic: washing machine room, kitchen, and living room
	\item nature: sports field, creek, city park
	\item office: small office with people working, hallway in office building, meeting room with discussion.
	\item public: subway station, office cafeteria, restaurant at lunchtime
	\item street: traffic intersection, town square, cafe on public square
	\item transportation: subway, bus, taxi
	\end{itemize}

\item Signal: real sounds in the environment

\item Sound sources: multiple

\item DOI: \href{10.5281/zenodo.1227121}{https://zenodo.org/record/1227121} 

\item Files available
	\begin{itemize}
	\item raw recordings in wav format (300ms)
	\item documentation
	\end{itemize}

\end{itemize}

\bibliography{references}
\bibliographystyle{plain}

\end{document}
