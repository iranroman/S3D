\documentclass[14pt, oneside]{extarticle} 
\usepackage{authblk, helvet, amsmath, amsthm, amssymb, calrsfs, wasysym, verbatim, bbm, color, graphics, geometry}
\renewcommand{\familydefault}{\sfdefault}

\geometry{tmargin=.75in, bmargin=.75in, lmargin=.75in, rmargin = .75in}  

\newcommand{\R}{\mathbb{R}}
\newcommand{\C}{\mathbb{C}}
\newcommand{\Z}{\mathbb{Z}}
\newcommand{\N}{\mathbb{N}}
\newcommand{\Q}{\mathbb{Q}}
\newcommand{\Cdot}{\boldsymbol{\cdot}}

\newtheorem{thm}{Theorem}
\newtheorem{defn}{Definition}
\newtheorem{conv}{Convention}
\newtheorem{rem}{Remark}
\newtheorem{lem}{Lemma}
\newtheorem{cor}{Corollary}


\title{A review of existings datasets with microphone array recordings}
\author{Iran R. Roman \thanks{roman@nyu.edu}}
\affil{Music and Audio Research Laboratory, New York University}
\date{March 2021}

\begin{document}

\maketitle
\tableofcontents

\vspace{.25in}

\section{Introduction}

\subsection{Background}

\begin{itemize}

\item Localization and tracking allow intelligent agents to determine if objects are moving, in which direction, and how fast. 

\item From an evolutionary standpoint, object localization and tracking are essential for survival \cite{heffner2018evolution}.

\item Sound localization and tracking are fundamental problems of machine listening technologies.

\item Without these capabilities, {\it listening machines} cannot locate living beings and objects in their environment by the sounds they make. 

\end{itemize}

\subsection{Problem statement}

\begin{itemize}

\item Current machine listening models learn sound localization and tracking using a specific microphone array configuration. 

\item To add or remove channels in the data, the model architecture must be changed and retrained, at least at the level of the input layer.

\end{itemize}

\subsection{Proposed solution}

\begin{itemize}

\item Microphone arrays come in all sizes and shapes \cite{kurz2015comparison, bates2017comparing, lopez2019sphear}. 

\item We should not think of individual microphone array configurations as mutually exclusive.

\item Instead, we can think of them as a subset of a virtual with an infinite number of channels.

\item The virtual array is spherical in shape, and contains microphones at all positions and facing all possible directions of arrival.

\item For each microphone in an existing array, we know its signal captured, position, directionality, and frequency response. 

\item By combining information across microphone arrays, we can interpolate and approximate the virtual array.

\end{itemize}

\section{Scope of this review}

\begin{itemize}

\item This document has two goals:
	\begin{itemize}
	\item To identify existing audio datasets collected with microphone arrays. 
	\item To discuss whether they can be used to approximate the virtual array. 
	\end{itemize}

\item First, we describe the importance of microphone arrays for object localization and tracking.

\item Then, we review the technical specifications of commonly used microphone arrays.

\item Next, we present an overview of existing datasets with microphone arrays.

\item Finally, we discuss methodologies to exploit these datasets with the goal of approximating the virtual array.  

\end{itemize}

\section{Object localization and tracking with microphone arrays}
\subsection{Auditory object localization and tracking}

\begin{itemize}

\item Intelligent organisms rely on sensory cues to interact with objects in their environment. 

\item Several mammalian species can localize and track objects based on the sounds they make.

\item To do this, many animals rely on their ears (two auditory inputs) and mechanisms of \cite{grothe2010mechanisms}:
	\begin{itemize}
	\item Monoaural cues (spectral analysis)
	\item Interaural time difference
	\item Interaural level difference
	\end{itemize}

\item Evolution also gave rise to other mechanisms for sound localization and tracking, including \cite{grothe2010mechanisms}: 
	\begin{itemize}
	\item Particle motion detection in fish 
	\item Processing of interfering patterns reaching the tympanum in frogs and birds
	\end{itemize}

\end{itemize}

\subsection{The need for microphone arrays}

\begin{itemize}

\item Many mammals are able to localize and track auditory objects in their enviroment using only two ears.

\item While a microphone array can have just two microphones, most have four or more. 

\item Why do we need microphone arrays to carry out the same task that many animals carry out with two ears?

\item There two main technical problems with binaural source localization in machines \cite{kim2015improved}:
	\begin{itemize}
	\item Diffraction of sound waves around the agent's head
	\item Front back ambiguity
	\end{itemize}

\item To overcome these issues, it is necessary to move the ears and head (like humans naturally do) and filter inputs with specific head-related transfer functions.

\item In other words, the machine has to listen more like a biological agent.
	
\item From an engineering stand-point, the issue can be solved using microphone arrays, which add spatial and auditory information. 

\item In machine listening, microphone arrays have been shown to be the optimal solution to carrying out sond source localization and tracking.

\item In general, the larger the number of microphones in an array, the better sound source localization results \cite{valin2007robust}.

\end{itemize}

\section{Commonly used microphone arrays}

\section{Existing datasets with microphone arrays}

\begin{itemize}

\subsection{Lee and Johnson, 2019}

\item 3D sound recordings of musical performances and room impulse responses.

\item Venue: St. Paul's concert hall (Huddersfield, UK)

\item Microphone arrays used simultaneously:
	\begin{itemize}
	\item OCT-3D
	\item 2L-Cube
	\item PCMA-3D
	\item Decca Tree
	\item Hamasaki Square
	\end{itemize}
	
\item Sound sources:
	\begin{itemize}
	\item String quartet
	\item Piano trio
	\item Piano solo
	\item Organ
	\item Clarinet solo
	\item Vocal group
	\item Room impulse response (virtual; 13 source positions)
	\end{itemize}

\item 250+ different front-rear-height combinations

\item \textbacklash 10 hours of data 

\item URL: \href{http://www.hud.ac.uk/apl/resources}{http://www.hud.ac.uk/apl/resources}

\end{itemize}

\bibliography{references}
\bibliographystyle{plain}

\end{document}
