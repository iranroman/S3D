\documentclass[14pt]{extarticle} 
\usepackage{authblk, helvet, amsmath, amsthm, amssymb, calrsfs, wasysym, verbatim, bbm, color, graphics, geometry, hyperref, lscape, makecell, longtable}
\usepackage[binary-units=true]{siunitx}
\renewcommand{\familydefault}{\sfdefault}

\geometry{tmargin=.75in, bmargin=.75in, lmargin=.75in, rmargin = .75in}  

\newcommand{\R}{\mathbb{R}}
\newcommand{\C}{\mathbb{C}}
\newcommand{\Z}{\mathbb{Z}}
\newcommand{\N}{\mathbb{N}}
\newcommand{\Q}{\mathbb{Q}}
\newcommand{\Cdot}{\boldsymbol{\cdot}}

\newtheorem{thm}{Theorem}
\newtheorem{defn}{Definition}
\newtheorem{conv}{Convention}
\newtheorem{rem}{Remark}
\newtheorem{lem}{Lemma}
\newtheorem{cor}{Corollary}


\title{Virtual microphone array: theoretical aspects}
\author{Iran R. Roman \thanks{roman@nyu.edu}}
\affil{Music and Audio Research Laboratory, New York University}
\date{}

\begin{document}

\maketitle
\tableofcontents

\vspace{.25in}

\section{Theoretical framework}

\begin{enumerate}

\item Consider an audio signal $a(t)$ and an environment $h$ with LTI impulse response $h_c(t)$, where $c$ denotes the cartesian coordinates $c=(x,y,z)$ where the impulse respose $h_c(t)$ exists.

\item $b_c(t) = (a(t) \ast h_c(t))$ is the convolution between the audio signal and the environment's impulse response at location $c=(x,y,z)$. 

\item No also consider a microphone $u$ with impulse response $u(t)$. Measuring signal $a(t)$ at location $c$ in the environment $h$ will result in recording $r_c(t) = (u(t) \ast b_c(t))$.

\item Given $r_c(t)$, one can obtain the time-delayed version $r_c(t-T)$ via convolution with a dirac delta function $\delta(T)$. So $r_c(t-T) = (r_c(t) \ast \delta(T))$.

\item The virtual array will consist of $N$ identical microphones $u$ (or $N$ microphones with different impulse responses) physically distributed throughout the environment.

\item All microphones in the virtual array listen a common signal $a(t)$, but at different locations. The measurement of the signal $a(t)$ in room $h$ by microphone $u_n$ at location $c$ is given by $r_{(n,c_n)}(t) = (u_n(t) \ast b_{c_n}(t)) = \big(u_n(t) \ast (a(t) \ast h_{c_n}(t))\big)$.

\item Microphones in the virtual array may differ in position $c$ and impulse response $u(t)$. The different locations will change the room impulse response $h_c(t)$ that each microphone reads. Moreover, since the signal $a(t)$ travels at the speed of sound, the version of $a(t)$ reaching a specific microphone will be delayed with respect to other microphones. However, as we already discussed, delays can be easly accounted for with delta functions. 

\item Because the microphones in the virtual array share a common signal $a(t)$, there exists an idealized transfer function $f(x)$ that can convert $r_{(n,c_n)}(t)$ into $r_{(m,c_m)}(t) = r_{(n,c_n)(t)} \ast f(r_{(n,c_n)},c_m)$. 

\item Such idealized transfer function does not exist in practice. However, a transfer function $h(x)$ can convert $r_{(n,c_n)}(t)$ into $r_{(m,c_m)}(t)$ with error $r_{(m,c_m)}(t) = r_{(n,c_n)} \ast h(r_{(n,c_n)}) + \epsilon$.

\item The question here is, how do we approximate $h(x)$ ?


\end{enumerate}


\bibliography{references}
\bibliographystyle{plain}

\end{document}
