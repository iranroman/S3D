\documentclass[14pt]{extarticle} 
\usepackage{authblk, helvet, amsmath, amsthm, amssymb, calrsfs, wasysym, verbatim, bbm, color, graphicx, geometry, hyperref, lscape, makecell, longtable}
\usepackage[binary-units=true]{siunitx}
\renewcommand{\familydefault}{\sfdefault}

\geometry{tmargin=.75in, bmargin=.75in, lmargin=.75in, rmargin = .75in}  

\newcommand{\R}{\mathbb{R}}
\newcommand{\C}{\mathbb{C}}
\newcommand{\Z}{\mathbb{Z}}
\newcommand{\N}{\mathbb{N}}
\newcommand{\Q}{\mathbb{Q}}
\newcommand{\Cdot}{\boldsymbol{\cdot}}

\newtheorem{thm}{Theorem}
\newtheorem{defn}{Definition}
\newtheorem{conv}{Convention}
\newtheorem{rem}{Remark}
\newtheorem{lem}{Lemma}
\newtheorem{cor}{Corollary}


\title{Virtual microphone array: theoretical aspects}
\author{Iran R. Roman \thanks{roman@nyu.edu}}
\affil{Music and Audio Research Laboratory, New York University}
\date{}

\begin{document}

\maketitle
\tableofcontents

\vspace{.25in}

\section{Introduction}

\subsection{Motivation}

\begin{itemize}

\item Microphone arrays have an intrinsic spatial structure. 

\item Learning such structure via self-supervised learning could result in a representation containing spatial information of sounds and environemtns recorded by the array.

\item This representation could be used to solve downstream tasks like sound source localization and  tracking.

\end{itemize}

\subsection{Problem statement}

\begin{itemize}

\item Current machine listening models learn sound localization and tracking using a specific microphone array configuration.

\item As a result, a unifying framework across models and microphone array configurations is  missing.

\item To change a model's microphone array or simply add or remove channels in the data, the model architecture must also be changed and retrained, at least at the level of the input-processing units.

\end{itemize}

\subsection{Proposed solution}

\begin{itemize}

\item Microphone arrays come in numerous sizes and shapes \cite{kurz2015comparison, bates2017comparing, lopez2019sphear}. 

\item We should not think of individual microphone array configurations as mutually exclusive.

\item Instead, we can think of them as a subset of a virtual array with an infinite number of channels.

\item The virtual array is spherical in shape, and contains microphones at all positions. 

\item By using information from existing individual microphones, we can approximate any microphone in the virtual array.

\end{itemize}

\section{Theoretical framework}

\begin{itemize}

\item Given signals recorded by a (group of) microphone(s), the main purpose of the virtual array is to use them to approximate the recording by any other arbitrary microphone in the virtual array. 

\end{itemize}

\begin{table}
	\centering
	\begin{tabular}{l|c}
	
	Description & Symbol \\
	
	\hline
	audio signal & $a(t)$ \\
	
	\hline
	environment & $h$ \\
	
	\hline
	cartesian coordinates & $c=(x,y,z)$ \\
	
	\hline
	impulse resp of $h$ at $c$ & $h_c(t)$ \\
	
	\hline 
	$a(t)$ at $h_c$ & $b_c(t) = a(t) \ast h_c(t)$ \\
	
	\hline
	microphone & $u_i$ \\
	
	\hline
	mic impulse resp & $u_i(t)$ \\
	
	\hline
	No. of microphones & $N$ \\
	
	\hline
	recording with $u_i$ at $h_c$ & $r_{(i,c)}(t) = (u_i(t) \ast b_c(t))$ \\ 
	
	\end{tabular}
	
	\caption{Variables needed to describe elements in the virtual microphone array}
	\label{tab:t1}
\end{table}

\subsection{Convolution operations in the virtual array}

\begin{itemize}

\item Consider an audio signal $a(t)$ and an environment $h$ with LTI impulse response $h_c(t)$.

\item $c$ denotes the cartesian coordinates $c=(x,y,z)$ where the impulse respose $h_c(t)$ exists.

\item The convolution between the audio signal and the environment's impulse response at location $c=(x,y,z)$ is: 

\begin{align}
b_c(t) = a(t) \ast h_c(t)
\end{align}

\item Also consider a microphone $u$ with impulse response $u(t)$. 

\item Measuring signal $a(t)$ at $h_c$ using microphone $u$ will result in recording: 

\begin{align}
r_c(t) = u(t) \ast b_c(t)
\end{align}

\item The virtual array will consist of $N$ microphones $u$ (microphones could have different impulse responses) physically distributed throughout the environment.

\item All microphones in the virtual array listen a common signal $a(t)$, but at different locations. 
\item The measurement of the signal $a(t)$ at $h_c$ by microphone $u_i$ is:

\begin{align*}
r_{(i,c)}(t) &= u_i(t) \ast b_{c}(t) \\
	&= u_i(t) \ast \big(a(t) \ast h_{c}(t)\big)
\end{align*}

\end{itemize}

\subsection{Generalized transformation between different $r_{(n,c)}$}

\begin{itemize}

\item Because the microphones in the virtual array share a common signal $a(t)$, there exists an idealized transfer function $f(x)$ that can convert $r_{(n,c_n)}(t)$ into $r_{(m,c_m)}(t)$ using $r_{n,c_n}$ and the spatial information $c_n$ and $c_m$:

\begin{align*}
r_{(m,c)}= r_{(n,c_n)}(t) \ast f(r_{(n,c_n)}(t),c_m)
\end{align*}

\item Such idealized transfer function does not exist in practice. However, a transfer function $l(x)$ can convert $r_{(n,c_n)}(t)$ into $r_{(m,c_m)}(t)$ with error:

\begin{align*}
r_{(m,c_m)}(t) = r_{(n,c_n)}(t) \ast l(r_{(n,c_n)}(t),c_m) + \epsilon
\end{align*}

\end{itemize}

\section{Methods to obtain $l(x)$}

\subsection{U-Net mask}

\begin{itemize}
		
\item Yang, Russell and Salamon \cite{yang2020telling} used a Spectral U-Net \cite{gao20192, stoller2018wave} to upmix stereo audio from mono. 

\begin{center}
\includegraphics[scale=0.65]{figures/yang_2020.png}
\end{center}

\item The virtual array is a similar problem.

\item Instead of a mono spectrogram, the input could be the spectrogram of multiple independent microphones listening the same signal in the same environment. 

\item The spatial information for each input microphone could be used to generate the weights of the first convolutional layer (using another network) \cite{simonovsky2017dynamic, xiong2015conditional}.

\item The spatial information for the desired microphone in the virtual array can be passed at the center of the U-Net.

\item The U-net output will be the same shape as the input, and can be used to multiply the original inputs and obtain the desired new channel. 

\end{itemize}

\subsection{DDSP}

\begin{itemize}

\item An autoencoder can learn embeddings from the time-frequency features of $r_{(n,c_n)}(t)$. 

\item 2D kernels will allow for learning of sound event detection and room acoustics. This is how the baseline method for the DCASE 2021 challenge task 3 \cite{Shimada2021}.

\item The embedding representation and spatial information $c_n  - c_m$ can be passed to a network that produces the parameters for a cascade of biquads that approximate $h(t)$.

\end{itemize}

\section{Delays due to the speed of sound}

\begin{itemize}

\item Given $r_c(t)$, one can also obtain the time-delayed version $r_c(t-T)$ via convolution with a dirac delta function $\delta(T)$. 

\begin{align*}
r_c(t-T) = r_c(t) \ast \delta(T)
\end{align*}

\item Microphones in the virtual array differ in position $c$. 

\item Since the signal $a(t)$ travels at the speed of sound, the version of $a(t)$ reaching a specific microphone will be delayed with respect to other microphones.

\item Due to relative location, microphone $u_m$ will receive the signal after (or before) microphone $u_n$. Because of this, forecasting (or inverse forecasting) at the head or tail of $r_{(c,c_n)}(t)$ will be a necessary step in the signal processing pipeline. 

\item Assuming a virtual array with a volume in the order of few meters, the delay between microphones in air will be in the order of few milliseconds (speed of sound in air is around \SI{343}{\metre\per\second}).

\item This delay of a few milliseconds could be accounted for by phase-shifts implicit in filtering. The only exception would be sharply transient events that could be completely missing in a recording, specially at the tail of the signal, or at the head in anechoicc onditions.

\item Existing approaches for audio signal forecasting include dilated convolutions \cite{oord2016wavenet} and transformers \cite{child2019generating}. 

\end{itemize}

\bibliography{references}
\bibliographystyle{plain}

\end{document}
